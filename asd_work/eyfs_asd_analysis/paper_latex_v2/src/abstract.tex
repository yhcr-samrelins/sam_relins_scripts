%! Author = samrelins
%! Date = 08/06/2022

We set out to replicate the work of Wright et al, demonstrating the link between early years fondation stage profile (EYFSP) scores and autism spectrum diagnosis, using a larger cohort taken from the Connected Bradford project. We conducated a retrospective data linkage study, using the post-2013 EYFSP scores from the Connected Bradford Department for Education data, and linked these with education census and primary care (GP records). The outcome measures were diagnosis of ASD using associated primary care (GP) practice codes (SNOMEDs). As with the Wright et al analysis, we used a ``total EYFSP score'' and the same subscore consisting of five key learning goals identiffierd using a panel of early-years autism experts.

This study validates the results of Wright et. al, and further demonstrates the fesibility of linking education and healthcare records using the Connected Bradford platform. A total of 70,277 children had linked primary care and education data, compared with a cohort of 8,935 children in the Wright et. al. study taken from Born In Bradford study. 17.3\% of the cohort scoring \textless 25 on the total EYFSP score, and 15.4\% scoring \textless 8 on the EYFSP subscore proposed in the Wright et al. analysis - individuals in these groups had a 11.09/16.5-fold increased odds of receiving an autism diagnosis respectively, when compared to baseline individuals scoring above these bounds.

\newpage
