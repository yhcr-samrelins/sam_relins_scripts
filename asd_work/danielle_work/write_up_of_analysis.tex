% Options for packages loaded elsewhere
\PassOptionsToPackage{unicode}{hyperref}
\PassOptionsToPackage{hyphens}{url}
%
\documentclass[
]{article}
\usepackage{amsmath,amssymb}
\usepackage{lmodern}
\usepackage{iftex}
\ifPDFTeX
  \usepackage[T1]{fontenc}
  \usepackage[utf8]{inputenc}
  \usepackage{textcomp} % provide euro and other symbols
\else % if luatex or xetex
  \usepackage{unicode-math}
  \defaultfontfeatures{Scale=MatchLowercase}
  \defaultfontfeatures[\rmfamily]{Ligatures=TeX,Scale=1}
\fi
% Use upquote if available, for straight quotes in verbatim environments
\IfFileExists{upquote.sty}{\usepackage{upquote}}{}
\IfFileExists{microtype.sty}{% use microtype if available
  \usepackage[]{microtype}
  \UseMicrotypeSet[protrusion]{basicmath} % disable protrusion for tt fonts
}{}
\makeatletter
\@ifundefined{KOMAClassName}{% if non-KOMA class
  \IfFileExists{parskip.sty}{%
    \usepackage{parskip}
  }{% else
    \setlength{\parindent}{0pt}
    \setlength{\parskip}{6pt plus 2pt minus 1pt}}
}{% if KOMA class
  \KOMAoptions{parskip=half}}
\makeatother
\usepackage{xcolor}
\usepackage[margin=1in]{geometry}
\usepackage{graphicx}
\makeatletter
\def\maxwidth{\ifdim\Gin@nat@width>\linewidth\linewidth\else\Gin@nat@width\fi}
\def\maxheight{\ifdim\Gin@nat@height>\textheight\textheight\else\Gin@nat@height\fi}
\makeatother
% Scale images if necessary, so that they will not overflow the page
% margins by default, and it is still possible to overwrite the defaults
% using explicit options in \includegraphics[width, height, ...]{}
\setkeys{Gin}{width=\maxwidth,height=\maxheight,keepaspectratio}
% Set default figure placement to htbp
\makeatletter
\def\fps@figure{htbp}
\makeatother
\setlength{\emergencystretch}{3em} % prevent overfull lines
\providecommand{\tightlist}{%
  \setlength{\itemsep}{0pt}\setlength{\parskip}{0pt}}
\setcounter{secnumdepth}{-\maxdimen} % remove section numbering
\usepackage{float} \usepackage{amsmath} \usepackage{booktabs} \usepackage{caption} \captionsetup{font=footnotesize}
\ifLuaTeX
  \usepackage{selnolig}  % disable illegal ligatures
\fi
\IfFileExists{bookmark.sty}{\usepackage{bookmark}}{\usepackage{hyperref}}
\IfFileExists{xurl.sty}{\usepackage{xurl}}{} % add URL line breaks if available
\urlstyle{same} % disable monospaced font for URLs
\hypersetup{
  pdftitle={Danielle/Leanne Analysis},
  pdfauthor={Sam Relins},
  hidelinks,
  pdfcreator={LaTeX via pandoc}}

\title{Danielle/Leanne Analysis}
\author{Sam Relins}
\date{2023-03-23}

\begin{document}
\maketitle

\hypertarget{notes}{%
\subsubsection{Notes:}\label{notes}}

\begin{itemize}
\tightlist
\item
  Regression on Age at Diagnosis: I'm leaving this out! Two main reasons
  - 1. I think the descriptive analysis above says all you need to say
  about the relationships between the variables we're looking at, and
  the regression data adds nothing 2. (more importantly!) we've shown
  above that age at diagnosis is bimodal for the whites group but
  unimodal for asians - simple gaussian regression assumes a unimodal
  outcome variable and certainly isn't well suited to a comparison of
  the covariates in this case. If we're desperate to come up with some
  sort of model for the data, we'd need to discuss a bit more, and it
  would be considerably more complex as a piece of work
\end{itemize}

\hypertarget{methods}{%
\section{Methods}\label{methods}}

This study explores the demographic features of individuals 18 years old
or younger, identified as having an Autism Spectrum Disorder (ASD) in
the Connected Bradford primary care dataset. The primary cohort is
defined as any individual with one or more SNOMED read-codes for ASD
diagnoses in their primary care record. Information on age, ethnic
group, are taken from the primary care record and form the basis of this
analysis. An ``age at diagnosis'' variable is defined as the date at
which the first ASD SNOMED code appears on an individual's records. The
vast majority of the cohort fall either in the ``White'' or ``Asian''
ethnic groups, so the other ethnicities have been grouped into an ``All
Other'' ethnic group as low numbers of records prevent meaningful
analysis of their subgroupings.

The likelihood of diagnosis, based on the demographic features in the
ASD cohort, is also described. The cohort for this analysis is defined
as individuals 18 years old or younger appearing in the connected
bradford education census data, and the outcome variable is a diagnosis
of ASD as defined by individuals from the census data that also appear
in the ASD cohort. Sex and ethnic group observations are taken from the
primary care data or, where absent, are determined from the data in the
education census records. Adjusted odds of ASD diagnosis and confidence
intervals for each of the demographic variables are calculated using
logistic regression.

\hypertarget{results}{%
\section{Results}\label{results}}

\begin{table}[H]
\centering
\begin{scriptsize}
\begin{tabular}{llll}
\toprule
            & & Missing & Count \\
\midrule
\textbf{n} & & & 2617 \\[2mm]
\textbf{Sex} & Male & 3 & 2022 (77.4) \\
& Female & & 592 (22.6) \\[2mm]
\textbf{Ethnicity} & White & 251 & 1547 (65.4) \\
            & Asian & & 636 (26.9) \\
            & Other & & 183 (7.7) \\[2mm]
\bottomrule
\end{tabular}
\end{scriptsize}
\caption{Sample charachteristics of ASD cohort}
\label{tab:asd_cohort}
\end{table}

Table \ref{tab:asd_cohort} shows the demographic breakdown of the ASD
cohort.
\textit{add comment here about the expected breakdown of ethnicities for this age group in the bradford population - I don't have a reference for this at the moment but it looks to me like there are fewer Asians than would be expected for under 18s - I can go on a hunt for this data if needed}.
The majority of the cohort is male (77.4\%)
\textit{insert comment about the accepted proportion of M/F with ASD - I'm assuming others will have a reference for this}

\begin{figure}[H]
\centering
\includegraphics[width=0.9\textwidth]{plots/age_hist.jpg}
\caption{Histogram showing counts of the age at diagnosis across the full ASD cohort}
\label{fig:age_hist}
\end{figure}

Figure \ref{fig:age_hist} shows the distribution of age at diagnosis -
it follows a bi-modal distribution with a sharp peak of diagnoses at age
4 and a shallower peak at age 9.
\textit{maybe some reference to causality for this if this phoenomena has already been studied - perhaps 2 different types of severity of ASD presentation that cause this pretty specific distribution}

\begin{figure}[H]
\centering
\includegraphics[width=0.7\textwidth]{plots/age_sex_box.jpg}
\caption{Box and whisker plots of age at diagnosis for males and females. The central line denotes the median, upper and lower bounds of the box denote the lower and upper quartiles. The "whiskers" (ends of lines protruding from each box) denote the highest/lowest data points that fall within 1.5 times the interquartile range above/below the bounds of the box.}
\label{fig:age_sex_box}
\end{figure}

\begin{figure}[H]
\centering
\includegraphics[width=0.9\textwidth]{plots/age_eth_box.jpg}
\caption{Box and whisker plots of age at diagnosis for each ethnicity group in the cohort. Produced using the same specification as Figure \ref{fig:age_sex_box}. Outliers (individual points plotted above/below the whiskers) are any data points that fall outside the 1.5 times interquartile range of the whiskers.}
\label{fig:age_eth_box}
\end{figure}

Subdivision of age at diagnosis by sex shows little difference between
the two sexes, with a slightly higher age of diagnosis amongst females.
When divided by ethnicity, it can be seen that a much higher average age
of diagnosis and a broader range of ages can been seen among white
individuals, with the bulk of diagnoses occurring at much younger ages
among the other ethnic groups.

\begin{figure}[H]
\centering
\includegraphics[width=0.9\textwidth]{plots/age_eth_hist.jpg}
\caption{Histograms of age at diagnosis for the white and asian ethnicities}
\label{fig:age_eth_hist}
\end{figure}

Further comparison of the distribution of ages of diagnoses in the white
and asian ethnicities (Figure \ref{fig:age_eth_hist}) reveals markedly
different distributions. A distinct bimodal pattern of diagnosis ages is
observed in the whites group, with peaks in diagnosis at 4 years of age
and then at 9, whereas in the asian group a long tailed unimodal
distribution can be seen with a peak only at age 4.

\begin{figure}[H]
\centering
\includegraphics[width=0.9\textwidth]{plots/age_sex_eth_box.jpg}
\caption{Box plots of age at diagnosis for each ethnic group subdivided by sex}
\label{fig:age_sex_eth_box}
\end{figure}

Comparison of the sex differences in age at diagnosis between the
different ethnic groups (Figure \ref{fig:age_sex_eth_box}) shows that
females are diagnosed later on average in the white and other ethnic
groups, but are diagnosed considerably earlier among the asian group.

\begin{figure}[H]
\centering
\includegraphics[width=0.9\textwidth]{plots/asd_map.jpg}
\caption{A choropleth map of the residential addresses of the individuals in the ASD cohort with addresses that fall within a bradford census ward. Bounding geometries are the census ward areas.}
\label{fig:asd_map}
\end{figure}

Figure \ref{fig:asd_map} shows a choropleth map of the home residences
of each of the individuals in with an ASD diagnosis by census ward.
\textit{I don't really know the geography or geographic makeup of bradford well enough to make any intelligent comments here - I don't know weather the concentration of diagnoses in Keighley and the south east is worth pointing out?}

\hypertarget{likelihood-of-diagnosis}{%
\subsection{Likelihood of Diagnosis}\label{likelihood-of-diagnosis}}

\begin{table}[H]
\centering
\begin{scriptsize}
\begin{tabular}{llllll}
\toprule
            & & & \multicolumn{3}{l}{\textbf{Grouped by ASD Diagnosis, n(\%)}} \\[1mm]
            & & Missing & Overall & No & Yes \\
\midrule
\textbf{n} & {} & & 130778 & 128396 & 2382 \\[2mm]
\textbf{Sex} & Male & 322 & 67300 (51.6) &  65451 (51.1) & 1849 (77.6) \\
& Female & & 63156 (48.4) & 62623 (48.9) & 553 (22.4) \\[2mm]
\textbf{Ethnicity} & White & 16805 & 61982 (54.4) & 60560 (54.2) & 1422 (66.4) \\
            & Asian & & 42069 (36.9) & 41510 (37.1) & 559 (26.1) \\
            & Other & & 9922 (8.7) & 9760 (8.7) & 162 (7.6) \\[2mm]
\bottomrule
\end{tabular}
\end{scriptsize}
\caption{Sample charachteristics of the education census cohort broken down by ASD diagnoses}
\label{tab:census_cohort}
\end{table}

In order to compare the likelihood of ASD diagnosis between the
different demographic groups, a cohort of individuals under the age of
18 from the Connected Bradford education census data was established.
Table \ref{tab:census_cohort} shows the demographic breakdown of this
cohort, along with a comparison of individuals with/without an ASD
diagnosis. A much greater proportion of individuals with an ASD
diagnosis are male than in the general education census cohort, and a
much larger proportion of individuals diagnosed with ASD are white.

\begin{table}[H]
\centering
\begin{scriptsize}
\begin{tabular}{lllllll}
\toprule
\multicolumn{7}{c}{\textbf{Logistic Regression Results:}} \\[1mm]
\midrule\\
\multicolumn{2}{l}{\hspace{1cm}\textbf{Formula:}} & \multicolumn{5}{l}{\textit{$ASD Diagnosis \sim Male + White + Other$}} \\[1mm]
\multicolumn{2}{l}{\hspace{1cm}\textbf{Model:}} & Logit & \multicolumn{2}{l}{\textbf{Method:}} & MLE & \\[0.5mm]
\multicolumn{2}{l}{\hspace{1cm}\textbf{No. Observations:}} & 113811 & \multicolumn{2}{l}{\textbf{Df Residuals:}} & 113807 & \\[0.5mm]
\multicolumn{2}{l}{\hspace{1cm}\textbf{Df Model:}} & 3 & \multicolumn{2}{l}{\textbf{Pseudo R\textsuperscript{2}:}} & 0.036 & \\[0.5mm]
\multicolumn{2}{l}{\hspace{1cm}\textbf{Log-Likelihood:}} & -10244 & \multicolumn{2}{l}{\textbf{Ll-Null:}} & -10635 & \\[0.5mm]
\multicolumn{2}{l}{\hspace{1cm}\textbf{LLR p-value:}} & \multicolumn{5}{l}{0.00} \\\\
\midrule\\
& & \textbf{Odds (95\% CI)} & \textbf{Coef} & \textbf{Std Err} & \textbf{z} & \textbf{P\textless|z|} \\[1.5mm]
\textbf{Intercept} & & \multicolumn{1}{c}{-} & -5.07 & 0.058 & -87.43 & 0.000 \\[1.5mm]
\textbf{Gender}: & & & & & & \\
& Female (Ref) & & & & & \\
& Male & 3.36 (3.03, 3.71) & 1.21 & 0.05 & 23.37 & 0.000 \\
\textbf{Ethnicity}: & & & & & & \\
& Asian (Ref) & & & & & \\
& White & 1.71 (1.55, 1.89) & 0.54 & 0.05 & 10.65 & 0.000 \\
& Other & 1.21 (1.01, 1.45) & 0.19 & 0.09 & 2.16 & 0.031 \\\\
\bottomrule
\end{tabular}
\end{scriptsize}
\caption{Statistics of the logistic regression of ASD diagnosis on ethnicity and sex. The baseline or reference individual is an asian female. "White" and "Other" denote the alternative ethnicity categories from the asian baseline.}
\label{tab:log_reg_results}
\end{table}

Table \ref{tab:log_reg_results} shows the specification and results of a
logistic regression modelling the likelihood of ASD diagnosis as a
function of Sex (male yes/no) and ethnicity (White yes/no, Other
yes/no). With comparison to a baseline individual who is female and
asian, a male is 3.36 (95\% CI 3.03, 3.71) times as likely to receive an
ASD diagnosis, a white individual is 1.71 (95\% CI 1.55, 1.89) times as
likely to be diagnosed, and an individual of other ethnicity is 1.21
times (95\% CI 1.01, 1.45) as likely to be diagnosed.

\hypertarget{limitations}{%
\subsection{Limitations}\label{limitations}}

This study uses an individual's primary care record to confirm both a
diagnosis of ASD, and the time at which a diagnosis was made. ASD
assessments are conducted by secondary care mental health services, and
thus using primary care as the main source of data in this study, it is
assumed that clinical records are being communicated and recorded
accurately between these different practices.

The assumption that an ASD clinical code denotes a diagnosis of ASD
implies that the absence of an ASD code means an individual does not
have a diagnosis. Many potential issues with the communication and
recording of diagnoses in an individuals primary care record could
result in diagnoses being recorded late or being absent all together.
This would affect the reported results both for the features of the
population diagnosed with ASD and the age at which these diagnoses
occur. The ASD cohort and the time of diagnosis data have yet to be
validated against the services that manage diagnoses, and so it is not
possible to identify any such issues with the cohort used in this study.

\end{document}
