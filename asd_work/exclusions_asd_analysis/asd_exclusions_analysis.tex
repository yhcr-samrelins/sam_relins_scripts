% Options for packages loaded elsewhere
\PassOptionsToPackage{unicode}{hyperref}
\PassOptionsToPackage{hyphens}{url}
%
\documentclass[
]{article}
\usepackage{amsmath,amssymb}
\usepackage{lmodern}
\usepackage{iftex}
\ifPDFTeX
  \usepackage[T1]{fontenc}
  \usepackage[utf8]{inputenc}
  \usepackage{textcomp} % provide euro and other symbols
\else % if luatex or xetex
  \usepackage{unicode-math}
  \defaultfontfeatures{Scale=MatchLowercase}
  \defaultfontfeatures[\rmfamily]{Ligatures=TeX,Scale=1}
\fi
% Use upquote if available, for straight quotes in verbatim environments
\IfFileExists{upquote.sty}{\usepackage{upquote}}{}
\IfFileExists{microtype.sty}{% use microtype if available
  \usepackage[]{microtype}
  \UseMicrotypeSet[protrusion]{basicmath} % disable protrusion for tt fonts
}{}
\makeatletter
\@ifundefined{KOMAClassName}{% if non-KOMA class
  \IfFileExists{parskip.sty}{%
    \usepackage{parskip}
  }{% else
    \setlength{\parindent}{0pt}
    \setlength{\parskip}{6pt plus 2pt minus 1pt}}
}{% if KOMA class
  \KOMAoptions{parskip=half}}
\makeatother
\usepackage{xcolor}
\usepackage[margin=1in]{geometry}
\usepackage{graphicx}
\makeatletter
\def\maxwidth{\ifdim\Gin@nat@width>\linewidth\linewidth\else\Gin@nat@width\fi}
\def\maxheight{\ifdim\Gin@nat@height>\textheight\textheight\else\Gin@nat@height\fi}
\makeatother
% Scale images if necessary, so that they will not overflow the page
% margins by default, and it is still possible to overwrite the defaults
% using explicit options in \includegraphics[width, height, ...]{}
\setkeys{Gin}{width=\maxwidth,height=\maxheight,keepaspectratio}
% Set default figure placement to htbp
\makeatletter
\def\fps@figure{htbp}
\makeatother
\setlength{\emergencystretch}{3em} % prevent overfull lines
\providecommand{\tightlist}{%
  \setlength{\itemsep}{0pt}\setlength{\parskip}{0pt}}
\setcounter{secnumdepth}{-\maxdimen} % remove section numbering
\usepackage{float} \usepackage{amsmath} \usepackage{booktabs} \usepackage{caption} \captionsetup{font=footnotesize}
\ifLuaTeX
  \usepackage{selnolig}  % disable illegal ligatures
\fi
\IfFileExists{bookmark.sty}{\usepackage{bookmark}}{\usepackage{hyperref}}
\IfFileExists{xurl.sty}{\usepackage{xurl}}{} % add URL line breaks if available
\urlstyle{same} % disable monospaced font for URLs
\hypersetup{
  pdftitle={ASD Exclusions Analysis},
  pdfauthor={Sam Relins},
  hidelinks,
  pdfcreator={LaTeX via pandoc}}

\title{ASD Exclusions Analysis}
\author{Sam Relins}
\date{2023-03-23}

\begin{document}
\maketitle

\hypertarget{analysis-methods}{%
\section{Analysis \& Methods}\label{analysis-methods}}

This study aims to explore the association between ASD diagnoses and
school exclusions and, in particular, any differences in rates of
exclusion between individuals with diagnosed and undiagnosed cases of
ASD. This was a retrospective data linkage study, using a cohort
collated from the Connected Bradford Whole System Data Linkage
Accelerator. The data used are taken from two primary sources: primary
care data from the 86 general practices in the Bradford area for ADD
OBSERVATION PERIOD, and Department for Education (DfE) data covering the
school census and reported exclusions for ADD OBSERVATION PERIOD.
Inclusion criteria were individuals appearing in both the DfE census and
the Bradford Primary Care datasets, as these individuals can be
reasonably assumed to have been under observation for both diagnoses of
Autism Spectrum Disorders and exclusions in any academic year.

Units of observation were the individual acadademic years of compulsory
education (reception to year 11) for each person appearing in the
education census data. ASD diagnoses are defined by the appearance one
or more SNOMED read-codes for ASD diagnoses in their primary care
record. A diagnosis date is estimated by the date at which the first ASD
SNOMED code appears on an individual's records. This diagnosis date is
used to define a ``pending'' or ``diagnosed'' variable for each academic
year. A confirmed diagnosis is a diagnosis date of ASD 6 months or more
prior to the start date of the academic year in question, otherwise the
diagnosis is recorded as pending - this six month figure is used to
introduce period of time for schools to have made adjustments to
accommodate an individual's diagnosis. The outcome of interest, an
``excluded'' variable, is defined as the presence of one or more records
of an exclusion (temporary or permanent) for an individual in a given
academic year.

Logistic regression was used to estimate the affect of ASD diagnoses on
the likelihood of exclusion. Individual persons were the unit of
observation for the regression analyses, aggregated over the years of
education for each individual. Given a clear distinction between the
patterns of exclusison in the primary and secondary levels of education
(Figure XXXX), two separate models of exclusion were proposed. A
``primary school'' model aggregated for each person over all the school
years Reception to year 6, and a ``secondary school model'' aggregated
over the years 7-11. A binary indicator of exclusions (Excluded/Not
Excluded) was modelled as a function of a person's ``diagnosis
status'\,' for ASD: No diagnosis (reference category), pending diagnosis
and confirmed diagnosis. The aggregated observations for the logistic
regression were defined as follows:

\begin{itemize}
\tightlist
\item
  \textbf{Excluded}: One or more exclusions in the academic years
  aggregated
\item
  \textbf{No Diagnosis}: No diagnosis of ASD in an individual's primary
  care record
\item
  \textbf{Pending Diagnosis}: A diagnosis status of pending in the
  academic year in which an exclusion is observed, or a diagnosis status
  of pending for all the academic years aggregated if no exclusions are
  observed,.
\item
  \textbf{Confirmed Diagnosis}: A diagnosis status of confirmed in any
  of the academic years aggregated, excluding examples where an
  exclusion was observed when a diagnosis was pending.
\end{itemize}

A significant proportion of the individuals in the cohort had missing
data for one or more of the years of their compulsory education
(reception to year 11). Analyses of these missing observations suggested
that these data were not missing at random, and that a dependence
relationship existed between these missing years of eduation and the
viariables of interest, ASD diagnoses and exclusions. To account for
this relationship, a further set of inclusion criteria were introduced:
only individuals with all available years of education at either the
primary, secondary or both levels of education were included in the
cohort. Further discussion of missing data, including a description of
the dependence relationships between missing data and the variables of
interest, can be found in the supplimentary material.

\hypertarget{results}{%
\section{Results}\label{results}}

Table \ref{fig:table_1} and Figure \ref{fig:excl_pcts} show the total
counts and proportions of exclusions and ASD diagnoses by academic year.
There is a clear change in the pattern of exclusions between the years
of primary education, reception to year 6, and secondary education,
years 7 to 11. Exclusions in primary education follow a slowly
increasing trajectory, in secondary education the rate of exclusion
increases sharply reaching a peak in year 10 and then falling in year
11. As expected, autism diagnoses increase over the years of education
with a corresponding decline in pending diagnoses. The proportions of
pending and diagnosed ASD cases fall from year 6 to year 7 on the basis
of the inclusion criteria for the cohort, see SUPPLIMENTARY MATERIAL for
further information.

\begin{table}[H]
\centering
\renewcommand*{\arraystretch}{1.2}
\begin{scriptsize}\begin{tabular}{llllll}
\\
\toprule
\textbf{Academic Year} & \textbf{Total} & \multicolumn{3}{l}{\textbf{ASD Diagnosis (\%)}} & \textbf{Excluded (\%)} \\
 & & None & Pending & Diagnosed & \\
\midrule
Reception & 111958 & 109991 (98.24) & 1853 (1.66) & 114 (0.10) & 61 (0.05) \\
Year 1 & 111835 & 109896 (98.27) & 1731 (1.55) & 208 (0.19) & 152 (0.14) \\
Year 2 & 111710 & 109774 (98.27) & 1626 (1.46) & 310 (0.28) & 222 (0.20) \\
Year 3 & 111613 & 109686 (98.27) & 1523 (1.36) & 404 (0.36) & 468 (0.42) \\
Year 4 & 111637 & 109704 (98.27) & 1416 (1.27) & 517 (0.46) & 725 (0.65) \\
Year 5 & 111635 & 109695 (98.26) & 1304 (1.17) & 636 (0.57) & 1078 (0.97) \\
Year 6 & 111676 & 109738 (98.26) & 1186 (1.06) & 752 (0.67) & 1342 (1.20) \\
Year 7 & 129538 & 127996 (98.81) & 1054 (0.81) & 488 (0.38) & 3116 (2.41) \\
Year 8 & 129630 & 128092 (98.81) & 940 (0.73) & 598 (0.46) & 5449 (4.20) \\
Year 9 & 129489 & 127949 (98.81) & 857 (0.66) & 683 (0.53) & 7670 (5.92) \\
Year 10 & 129743 & 128207 (98.82) & 734 (0.57) & 802 (0.62) & 9424 (7.26) \\
Year 11 & 129602 & 128058 (98.81) & 642 (0.50) & 902 (0.70) & 6774 (5.23) \\
\bottomrule
\end{tabular}\end{scriptsize}
\caption{Counts and percentages for the cohort by academic year. Percentages for ASD diagnoses and exclusions are reported as a percentage of the entire cohort.}
\label{fig:table_1}
\end{table}

\begin{figure}[H]
\centering
\includegraphics[width=0.9\textwidth]{plots/asd_excl_pcts.jpg}
\caption{Proportions of the cohort that have an exclusion, an ASD diagnosis or a pending a diagnosis by academic year. Proportions are given as percentages of the whole cohort}
\label{fig:excl_pcts}
\end{figure}

\begin{figure}[H]
\centering
\includegraphics[width=0.9\textwidth]{plots/excl_pcts_grouped.jpg}
\caption{Proportions of cohort with an exclusion by academic year subdivided into ASD diagnosis status: No diagnosis (``None''), already diagnosed, or pending a diagnosis.}
\label{fig:excl_pcts}
\end{figure}

The trend of exclusions across the academic years can be seen in Figure
\ref{fig:excl_pcts}, subdivided by ASD diagnosis status. Students with
pending diagnoses had a consistently higher rate of exclusion than those
without an ASD diagnosis or a confirmed diagnosis of an ASD. Students
with an ASD diagnosis had higher rates of exclusion than those with no
diagnosis in primary school and up to year 8, and then lower rates of
exclusion than thereafter.

\begin{table}[H]
\centering
\renewcommand*{\arraystretch}{1.2}
\begin{scriptsize}\begin{tabular}{lllll}
\toprule
& \textbf{ASD Diagnosis} & \multicolumn{2}{l}{\textbf{Excluded (\%):}} & \textbf{Total} \\
& & No & Yes & \\
\midrule
\textbf{Primary}: & & & & \\
& None & 107171 (97.69) & 2536 (2.31) & 109707 \\
& Pending & 1078 (88.14) & 145 (11.86) & 1223 \\
& Diagnosed & 682 (95.79) & 30 (4.21) & 712 \\[1mm]
& \textbf{Total:} & 108931 (97.57) & 2711 (2.43) & 111642 \\
\textbf{Secondary}: & & & & \\
& None & 108190 (84.51) & 19827 (15.49) & 128017 \\
& Pending & 523 (74.82) & 176 (25.18) & 699 \\
& Diagnosed & 756 (90.32) & 81 (9.68) & 837 \\[1mm]
& \textbf{Total:} & 109469 (84.50) & 20084 (15.50) & 129553 \\
\bottomrule
\end{tabular}\end{scriptsize}
\caption{Counts from aggregated primary (Reception to year 6) and sedcondary (year 7 to year 11) datasets subdivided by ASD diagnosis status and exclusions. Proportions are reported as those observed as excluded yes/no as a percentage of the ASD diagnosis subgroup.}
\label{tab:table_2}
\end{table}

Aggregated data for primary and secondary years can be seen in Table
\ref{tab:table_2}. Overall rates of exclusion are higher in Secondary
school (15.5\% of students excluded) than in primary (2.43\% of students
excluded) as seen in the yearly data. Students with pending ASD
diagnoses are excluded at the highest rates in both primary and
secondary years (11.86\% and 25.18\% respectively). Students with a
confirmed diagnosis of an ASD are excluded at a higher rate than those
without an ASD diagnosis in primary school (4.21\% compared with
2.31\%), but are excluded at a lower rate than those without a diagnosis
in secondary school (9.68\% compared with 15.49\%).

\begin{table}[H]
\centering
\begin{scriptsize}
\begin{tabular}{lllllll}
\toprule
\multicolumn{7}{c}{\textbf{Logistic Regression Results:}} \\[1mm]
\midrule\\
\multicolumn{2}{l}{\hspace{1cm}\textbf{Formula:}} & \multicolumn{5}{l}{\textit{$Excluded \sim Pending Diagnosis + Diagnosed ASD$}} \\[1mm]
\multicolumn{2}{l}{\hspace{1cm}\textbf{Model:}} & Logit & \multicolumn{2}{l}{\textbf{Method:}} & MLE & \\[1.5mm]
\multicolumn{2}{l}{\textbf{Primary:}} & & & & & \\[1.5mm]
\multicolumn{2}{l}{\hspace{1cm}\textbf{No. Observations:}} & 113811 & \multicolumn{2}{l}{\textbf{Df Residuals:}} & 113807 & \\[0.5mm]
\multicolumn{2}{l}{\hspace{1cm}\textbf{Df Model:}} & 3 & \multicolumn{2}{l}{\textbf{Pseudo R\textsuperscript{2}:}} & 0.036 & \\[0.5mm]
\multicolumn{2}{l}{\hspace{1cm}\textbf{Log-Likelihood:}} & -10244 & \multicolumn{2}{l}{\textbf{Ll-Null:}} & -10635 & \\[0.5mm]
\multicolumn{2}{l}{\hspace{1cm}\textbf{LLR p-value:}} & \multicolumn{5}{l}{0.00} \\\\
\multicolumn{2}{l}{\textbf{Secondary:}} & & & & & \\[1.5mm]
\multicolumn{2}{l}{\hspace{1cm}\textbf{No. Observations:}} & 113811 & \multicolumn{2}{l}{\textbf{Df Residuals:}} & 113807 & \\[0.5mm]
\multicolumn{2}{l}{\hspace{1cm}\textbf{Df Model:}} & 3 & \multicolumn{2}{l}{\textbf{Pseudo R\textsuperscript{2}:}} & 0.036 & \\[0.5mm]
\multicolumn{2}{l}{\hspace{1cm}\textbf{Log-Likelihood:}} & -10244 & \multicolumn{2}{l}{\textbf{Ll-Null:}} & -10635 & \\[0.5mm]
\multicolumn{2}{l}{\hspace{1cm}\textbf{LLR p-value:}} & \multicolumn{5}{l}{0.00} \\\\
\midrule\\
& & \textbf{Odds (95\% CI)} & \textbf{Coef} & \textbf{Std Err} & \textbf{z} & \textbf{P\textless|z|} \\[1.5mm]
\multicolumn{2}{l}{\textbf{Primary:}} & & & & & \\[1.5mm]
\multicolumn{2}{l}{\hspace{0.5cm}\textbf{Intercept}} & \multicolumn{1}{c}{-} & -5.07 & 0.058 & -87.43 & 0.000 \\[0.5mm]
\multicolumn{2}{l}{\hspace{0.5cm}\textbf{ASD Diagnosis:}} & & & & & \\
& \hspace{0.5cm}No Diagnosis (Ref) & & & & & \\
& \hspace{0.5cm}Pending Diagnosis & 3.36 (3.03, 3.71) & 1.21 & 0.05 & 23.37 & 0.000 \\
& \hspace{0.5cm}Diagnosed & 3.36 (3.03, 3.71) & 1.21 & 0.05 & 23.37 & 0.000 \\\\
\multicolumn{2}{l}{\textbf{Secondary:}} & & & & & \\[1.5mm]
\multicolumn{2}{l}{\hspace{0.5cm}\textbf{Intercept}} & \multicolumn{1}{c}{-} & -5.07 & 0.058 & -87.43 & 0.000 \\[0.5mm]
\multicolumn{2}{l}{\hspace{0.5cm}\textbf{ASD Diagnosis:}} & & & & & \\
& \hspace{0.5cm}No Diagnosis (Ref) & & & & & \\
& \hspace{0.5cm}Pending Diagnosis & 3.36 (3.03, 3.71) & 1.21 & 0.05 & 23.37 & 0.000 \\
& \hspace{0.5cm}Diagnosed & 3.36 (3.03, 3.71) & 1.21 & 0.05 & 23.37 & 0.000 \\\\
\bottomrule
\end{tabular}
\end{scriptsize}
\caption{Statistics of the logistic regression of ASD diagnosis on ethnicity and sex. The baseline or reference individual is an asian female. "White" and "Other" denote the alternative ethnicity categories from the asian baseline.}
\label{tab:log_reg_results}
\end{table}

\begin{table}[h]
\centering
\renewcommand*{\arraystretch}{1.2}
\begin{scriptsize}\begin{tabular}{lllc}
\toprule
& \textbf{Coefficient} & \textbf{Odds} & \textbf{95\% CI} \\
\midrule
\textbf{Primary:} & & & \\
& No Diagnosis & Baseline & - \\
& Pending Diagnosis & 5.68 & (4.76, 6.79) \\
& Diagnosed & 1.86 & (1.29, 2.69) \\ 
\textbf{Secondary:} & & & \\
& No Diagnosis & Baseline & - \\
& Pending Diagnosis & 1.84 & (1.54, 2.18) \\
& Diagnosed & 0.58 & (0.46, 0.74) \\ 
\bottomrule
\end{tabular}\end{scriptsize}
\caption{Odds ratios and confidence intervals of the effects of pending or confirmed ASD diagnoses with comparison to a baseline of no diagnosis of ASD, as estimated by the logistic regression model specified in the methods.}
\label{tab:model_summary}
\end{table}

At primary level, the regression modelling estimated that students with
a pending diagnosis or confirmed diagnosis of ASD have higher odds of
being excluded than students with no diagnosis at primary level (OR =
5.68 {[}95\%CI 4.76, 6.79{]} and OR = 1.86 {[}95\%CI 1.29, 2.69{]}
respectively). At secondary level, the regression models estimated that
students with a pending diagnosis again had higher odds of exclusion (OR
= 1.84 {[}95\%CI 1.54, 2.18{]}), however students with a confirmed
diagnosis had lower estimated odds of exclusion at secondary level (OR =
0.58 {[}95\%CI 0.46, 0.74{]})

\hypertarget{limitations}{%
\subsubsection{Limitations:}\label{limitations}}

\begin{itemize}
\tightlist
\item
  Those without ASD read codes or entries in Exclusions table assumed
  not to have ASD/been excluded - absence doesn't necessarily mean data
  isn't missing
\item
  inclusion criteria potentially bias those that are not excluded -
  excluded kids more likely to have missing records in census that lead
  to them being excluded from report
\item
  model poorly specified - comment on overall accuracy - need to
  consider confounders etc etc
\item
  more ASD diagnoses in recent years, fewer exclusions in recent years -
  this interacts with year group info - do exclusions increase
  controling for academic year?
\end{itemize}

\hypertarget{supplimentary-material}{%
\subsection{Supplimentary Material:}\label{supplimentary-material}}

List of ASD SNOMEDS - Link to OpenSafely ASD list!

\hypertarget{missing-data}{%
\subsection{Missing Data}\label{missing-data}}

See hand-written notes!

\begin{figure}[H]
\centering
\includegraphics[width=0.9\textwidth]{plots/missing_groups.jpg}
\caption{Proportion of missing observations by academic year for whole cohort before the additional inclusion critera that restrict the cohort to individuals with limited missing data. Data are subdivided by individuals that appear at any point in the exclusions data, and individuals that appear in the ASD cohort. Academic years that would have taken place before or after the period of observation are not considered missing in these data.}
\label{fig:missing_groups}
\end{figure}

\begin{figure}[H]
\centering
\includegraphics[width=0.9\textwidth]{plots/asd_props_all.jpg}
\caption{Proportions of ASD diagnoses by academic year for the whole cohort before the additional inclusion criteria that ristrict the cohort to individuals with limited missing data. Gray bars represent the total proportion of those with a recorded ASD diagnosis, colored lines the proportions of those diagnoses that are pending or confirmed in that academic year.}
\label{fig:asd_props_all}
\end{figure}

\begin{itemize}
\tightlist
\item
  ASD Diagnoses decrease as a proportion of overall data for each year
  as you go up academic years
\item
  Proportion of pending cases falls much faster than proportion of
  confirmed cases as you go up academic years - probably because kids
  diagnosed at a greater rate more recently
\item
  Pattern of missing data different between exclusions cohort, ASD
  cohort and census cohort
\item
  overall cohort slowly declines through primary school then peaks y7
  and declines back to baseline again - no real thoughts on this pattern
\item
  exclusions cohort declines smoothly to y-9 then increases sharply -
  exclusions are more frequent in data as you go up years so likelihood
  of you appearing in census increases as a functioin of increaseing
  academic years if you've been excluded - inverse pattern to exclusions
  over time in figure whatever
\item
  ASD missing data increses slowly across time - you're more likely to
\end{itemize}

\begin{figure}[H]
\centering
\includegraphics[width=0.9\textwidth]{plots/asd_props_sub.jpg}
\caption{Proportions of ASD diagnoses by academic year for the whole cohort after additional inclusion criteria that ristrict the cohort to individuals with limited missing data are imposed. Gray bars represent the total proportion of those with a recorded ASD diagnosis, colored lines the proportions of those diagnoses that are pending or confirmed in that academic year.}
\label{fig:asd_props_sub}
\end{figure}

\begin{itemize}
\tightlist
\item
  after inclusion criteria applied pattern of missingness v similar
  between ASD/excluded/rest
\item
  proportions of ASD/pending diagnoses much more sensible, total
  proportion of ASD remains stable across primary and secondary - sudden
  change in y7 is function of inclusion criteria
\end{itemize}

The patterns of missing data in the original cohort were examined,
focusing on the variables related to ASD diagnosis and school exclusion.
It was found that the proportion of missing data varied by academic year
and by the individuals with ASD diagnoses and/or a history of school
exclusion. Figure 1 shows the proportion of missing data by academic
year for individuals with an ASD diagnosis (pending or confirmed) and
for individuals with one or more exclusions on their academic records
(fixed-term or permanent exclusions).

The proportion of missing data decreases as a function of academic year
for all categories of ASD diagnosis, but the rate of decrease is
different for each category. The proportion of missing data for students
with no diagnosis is relatively stable across academic years, ranging
from 11.5\% in year 1 to 9.7\% in year 11. The proportion of missing
data for students with a pending diagnosis drops sharply from 25.6\% in
year 1 to 6.4\% in year 11. The proportion of missing data for students
with a confirmed diagnosis also decreases from 22.3\% in year 1 to
12.9\% in year 11, but less steeply than for students with a pending
diagnosis. These patterns suggest that ASD diagnoses are more likely to
be made or recorded in later academic years, especially for students
with a pending diagnosis.

As Figure 2 shows, the proportion of missing data also varies by
academic year and by exclusion category, but in a different way than for
ASD diagnosis. The proportion of missing data for students with no
exclusion is relatively constant across academic years, ranging from
12.3\% in year 1 to 10.4\% in year 11. The proportion of missing data
for students with at least one fixed-term exclusion decreases from
17.8\% in year 1 to 9.6\% in year 9, but then increases to 13.4\% in
year 11. The proportion of missing data for students with at least one
permanent exclusion also decreases from 23.4\% in year 1 to 10.7\% in
year 9, but then increases to 16.8\% in year 11. These patterns suggest
that school exclusions are more likely to occur or be recorded in later
academic years, especially for students with permanent exclusions.

The different patterns of missing data between individuals with
exclusions and individuals with ASD diagnoses indicate that the
mechanisms underlying the missingness may not be the same for these two
groups. Therefore, we need to account for the potential effects of
missing data on our analyses and use appropriate methods to handle them.
In the next section, we describe the methods we used to address the
missing data problem and test the robustness of our results.

One way to account for the missing data is to restrict the analysis to
individuals who have complete data for all the variables of interest.
Therefore, we applied additional inclusion criteria to our cohort that
removed any individuals who had one or more missing values from either
their primary school years (years 1 to 6) or secondary school years
(years 7 to 11). This resulted in a final sample size of N individuals,
with M individuals with ASD diagnosis and L individuals with exclusion.
Table 1 shows the descriptive statistics of the final sample by ASD
diagnosis and exclusion categories. This approach ensures that we have a
complete dataset for our analyses, but it also reduces the sample size
and may introduce selection bias if the missing data are not completely
random. In the next section, we describe how we assessed the impact of
this approach on our results and compared it with alternative methods to
handle the missing data.

\end{document}
